\section{TLS}
\textit{Transport Layer Security} (TLS), the successor of the now-deprecated \textit{Secure Sockets Layer} (SSL), is a cryptographic protocol designed to provide communications security over a computer network. Here is how it works:
\begin{enumerate}
    \item $A\rightarrow B$: \textit{I want to talk} ($m_1$, aka “client\_hello”)
    \item $B\rightarrow A$: Certificate ($m_2$, aka “server\_hello”)
    \item $A\rightarrow B$: ($m_3$ , aka “client key exchange” \& “finished”)
    \item $B\rightarrow A$: ($m_4$, “finished”)
\end{enumerate}
TLS is composed by two layers of protocols:
\begin{enumerate}
    \item Record Protocol
    \item Handshake, Change Cipher Spec, Alert, Heartbeat Protocols
\end{enumerate}
In order to understand how TLS operates, we need to learn two architecture concepts:
\begin{itemize}
    \item \textbf{TLS connection}: connections are transient, every connection is associated with one session (one session can have multiple connections). The connection state parameters are: server and client random, server write key, client write key, server write MAC secret, client write MAC secret, IV (CBC mode), sequence numbers.
    \item \textbf{TLS session}: an association between a client and a server, created by the Handshake Protocol. Define a set of cryptographic security parameters which can be shared among multiple connections (used to avoid the expensive negotiation of new security parameters for each connection).
    The session state parameters are: session identifier, peer certificate, compression method, cipher spec, master secret, is\_resumable
\end{itemize}
\subsection{Record Protocol}
The TLS Record Protocol provides two services for TLS connections
\begin{itemize}
    \item Confidentiality: the Handshake Protocol defines a shared secret key that is used for conventional encryption of TLS payloads
    \item Message integrity: the Handshake Protocol also defines a shared secret key that is used to form a message authentication code (MAC)
\end{itemize}
Client and server each maintain a set of keys for sending and receiving (plus potentially an IV)
\subsection{Heartbeat Protocol}
The Heartbeat Protocol is a periodic signal generated by hardware or software to indicate normal operation or to synchronize other parts of a system (typically used to monitor the availability of a protocol entity). It consists of two message types, \texttt{hb\_request} and \texttt{hb\_response}. The heartbeat serves two purposes
\begin{itemize}
    \item It assures the sender that the recipient is still alive
    \item The heartbeat generates activity across the connection during idle periods, which avoids closure by a firewall that does not tolerate idle connections
\end{itemize}
\subsection{HTTPS}
HTTPS refers to the combination of HTTP and SSL to implement secure communication between Web browser and server. When HTTPS is used, the following elements of the communication are encrypted:
\begin{itemize}
    \item URL of the requested document
    \item Contents of the document
    \item Contents of browser forms
    \item Cookies sent
    \item Contents of HTTP header
\end{itemize}