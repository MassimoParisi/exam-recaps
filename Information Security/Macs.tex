\section{Macs}
A Message Authentication Code (MAC) is a short piece of information used for authenticating a message. It uses a shared secret key to provide integrity and authenticity, but differently from symmetric encryption it is not reversible. To do so, we can use hash functions or symmetric block cipher.\\
We can use hashes as MAC, but we have to be cautious on constructing them: if we use MAC$=MD(K_{AB}||msg)$, someone could perform a \textit{length extension attack}:
\begin{itemize}
    \item Alice sends "message, MAC" to Bob
    \item an attacker can construct an extension MAC'$=MD(MAC||msg')=MD(K_{AB}||msg||msg')$
\end{itemize}
To avoid this, we can construct the MAC differently:
\begin{itemize}
    \item MAC$=MD(msg||K_{AB})$
    \item Use only half of $MD$ (not extensible)
    \item Use MAC$=MD(MD(K_{AB}||msg))$
\end{itemize}

\subsection{HMAC}
HMAC is the standard MAC used in IPSec. It can use different hash functions, and what it does is to compute $h(K_{AB}\xor a || h(K_{AB}\xor b ||m))$, where $a$ and $b$ are two constants. With this structure, even if the attacker found $m'$ such that $m\neq m'$ and $h(m)=h(m')$, he would still be unable to "swap out" $m$ due to the prefixed key and the double hash.

\subsection{CBC-MAC}
An alternative to HMAC is CBC-MAC, that is simply the last ciphertext block of CBC (with $IV=0$). CBC-MAC is only secure for fixed length messages, because with variable-length messages with the same key $k$, an attacker who knows the MACs of two different messages $m_1$ and $m_2$ can easily compute the MAC of a third message $m_3$, even if he does not know the key $k$.
\begin{itemize}
    \item Assume the attacker know two messages and their MACs $(m_1,t_1)$ and $(m_2,t_2)$
    \item They create $m_3=m_1||m_2^*$, where $m_2^*$ is simply $m_2$ with the first block XORed with $t_1$
    \item They know the MAC of $m_3$ must be $t_2$
\end{itemize}

\subsection{Digital signatures vs. MAC}
We have to remember that there is a clear difference between digital signatures and MACs: a digital signature provides \textit{non-repudiation}, while MAC provides \textit{authenticity}
