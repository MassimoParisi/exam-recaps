\section{Computer Security}
Access Control (AC) is the set of (limited) action a user can do in a system. We use AC to prevent security breaches. The main access rights to data and programs are \texttt{read}, \texttt{write} and \texttt{execute}.
\subsection{General AC Principles}
\begin{itemize}
    \item System must first authenticate a user seeking access
    \item Then, AC reference monitor determines if specific requested access by this user is permitted
    \item A security administrator maintains an authorization database
    \item An auditing function monitors and keeps a record of user accesses to system resources (accountability, flaws)
\end{itemize}
\subsection{ACLs vs. Capabilities}
The authorization database can be organized in different way:
\begin{itemize}
    \item Access Control List (ACL), often used in filesystems, contains a list for each \texttt{object} containing the right of each \texttt{subject} on it
    \item Capabilities is the dual of ACL. It contains a list for each \texttt{subject} containing the right on each \texttt{object}
\end{itemize}
\subsection{UNIX Permissions}
UNIX used ACLs and for each \texttt{object}, it indicated the rights of the \textit{owner}, the \textit{group owner} and the \textit{world}
\paragraph{SUID}
UNIX also offer the possibility for using \textbf{SUID} root, that allow users to run an executable with the filesystem permissions of the executable's owner or group owner. They are often used to allow users on a computer system to run programs with temporarily elevated privileges in order to perform a specific task. The introduction of SUID root led to several security vulnerability.
\paragraph{\texttt{sudo}} A better solution could be to use \texttt{sudo}, allowing selected users to temporarily become root (authenticating with their \texttt{pwd})

\subsection{Types of AC Policies}
\begin{itemize}
    \item \textbf{Discretionary AC}: security decisions up to object owner, different defaults option possible, no coherent security policy enforceable
    \item \textbf{Mandatory AC}: system-wide security policy enforcement, all objects and subjects have "security labels", access granted iff security labels "match", allows control of information flows
\end{itemize}


\subsubsection{Mandatory Access Policies Policy}
\paragraph{BLP Model}
The Bell-La Padula Model consists of two properties:
\begin{itemize}
    \item \textbf{Read Down}: a \texttt{subject}'s must dominate the security level of the \texttt{object} being read. More formally, $S$ can read $O$ iff $L_S \geq L_O \cap C_S \supseteq C_O$ and $S$ has discretionary read access to $O$
    \item \textbf{Write Up}: a \texttt{subject}'s clearance must be dominated by the security level of the \texttt{object} being written to. More formally, $S$ can write to $O$ iff $L_S \leq L_O \cap C_S \subseteq C_O$ and $S$ has discretionary write access to $O$
\end{itemize}

\paragraph{Problem} High clearance subjects can never communicate with low clearance subjects, so we need dynamic security levels. BLP protects confidentiality, but not integrity

\paragraph{Biba Model}
It is a state machine model to ensure integrity with these property:
\begin{itemize}
    \item \textbf{No Read Down}: a \texttt{object}'s integrity level must dominate the integrity level of the \texttt{subject} reading it ($I_O \geq I_S$)
    \item \textbf{No Write Up}: a \texttt{subject}'s integrity level must dominate the integrity level of the \texttt{object} being written to ($I_S \geq I_O$)
\end{itemize}